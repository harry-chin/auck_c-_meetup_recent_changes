% Created 2016-06-02 Thu 16:16
\documentclass[aspectratio=1610]{beamer}
\usepackage[utf8]{inputenc}
\usepackage[T1]{fontenc}
\usepackage{minted}
\usepackage{booktabs}
\usepackage{sourcecodepro}
%\usepackage{sourcesanspro}
\usepackage{arev}

\usetheme{Goettingen}
\usecolortheme{}
\usefonttheme[onlysmall]{structurebold}
\useinnertheme{}
\useoutertheme{}
\setbeamertemplate{navigation symbols}{}

\date{Auckland C++ Meetup\\14 June 2016}
\title{Recent Changes in C++}
\subtitle{Some Highlights}
\author[Toby Allsopp]{Toby Allsopp\\\texttt{toby@mi6.gen.nz}}

\hypersetup{
 pdfauthor={},
 pdftitle={},
 pdfkeywords={},
 pdfsubject={},
 pdfcreator={},
 pdflang={English}}

\begin{document}

%\usemintedstyle{murphy}
\setminted{frame=single}
\newminted{cpp}{autogobble}
\newmintinline[cpp]{cpp}{}

\frame{\titlepage}

\section{Introduction}

\begin{frame}
  \frametitle{This talk}
  \begin{itemize}
  \item Can only scratch the surface
  \item Focuses on what \emph{I} think is important
    \begin{itemize}
    \item stuff you should use in your own code
    \item stuff you need to know to read other code
    \end{itemize}
  \end{itemize}
\end{frame}

\begin{frame}
  \frametitle{Overview}
  \tableofcontents
\end{frame}

\begin{frame}
  \frametitle{History of the standard}
  \begin{itemize}
  \item C++98 was the first ISO standard
  \item C++03 had only minor tweaks
  \item C++11 was massive, 13 years since the last major update
  \item C++14 was much more modest
  \item C++17 also looks to be pretty modest
  \end{itemize}
\end{frame}

\section{Type deduction}

\begin{frame}[fragile]
  \frametitle{\cpp~auto~}
  Instead of
  \begin{cppcode}
    map<string, string>::iterator it = m.find("foo");
  \end{cppcode}
  you can write
\begin{minted}[]{c++}
auto it = m.find("foo");
\end{minted}
  \begin{itemize}
  \item No more \cpp~typedef std::map<string, string> FooBarMap~
  \item Doesn't work for member variables, function parameters (but wait for
    generic lambdas)
  \end{itemize}
\end{frame}

\begin{frame}[fragile]
  \frametitle{New function syntax}
  You can put the return type \emph{after} the parameter list.
\begin{minted}{cpp}
int f(double x) { return 7; }
double g(string s) { return 4.2; }

auto f(double x) -> int { return 7; }
auto g(string s) -> double { return 4.2; }
\end{minted}
  This is super useful when the return type of a function template depends on
  its parameters, e.g.
\begin{minted}{cpp}
template<typename L, typename R>
auto add(L l, R r) -> decltype(l + r) {
  return l + r;
}
\end{minted}
\end{frame}

\begin{frame}[fragile]
  \frametitle{Function return type deduction (C++14)}
  In C++14, you can leave out the return type entirely, e.g.
\begin{minted}{cpp}
template<typename L, typename R>
auto add(L l, R r) {
  return l + r;
}
\end{minted}
  You can use this for all your functions if you want as long as all callers can
  see the definition.
\end{frame}

\section{Lambda expressions}

\begin{frame}[fragile]
  \frametitle{Lambdas}
  Given
\begin{minted}[]{c++}
vector<int> v = { 1, 1, 2, 3 };
int limit = 1;
\end{minted}
  then
\begin{minted}[]{c++}
auto it = find_if(v.begin(), v.end(),
  [limit](int x) -> bool { return x > limit; });
\end{minted}
  is equivalent to
\begin{minted}[]{c++}
struct pred {
  int limit;
  pred(int limit) : limit(limit) {}
  bool operator()(int x) const { return x > limit; }
};
auto it = find_if(v.begin(), v.end(), pred(limit));
\end{minted}
\end{frame}

\begin{frame}[fragile]
  \frametitle{Lambda capturing spec}
  \begin{tabular}[h]{ll}
    \toprule
    \structure{C++11} \\
    Nothing                 & \cpp{[]}             \\
    Everything by reference & \cpp~[&]~            \\
    Everything by value     & \cpp~[=]~            \\
    Something by reference  & \cpp{[&something]}   \\
    Something by value      & \cpp{[something]}    \\
    One of each             & \cpp{[&byref,byval]} \\
    \midrule
    \structure{C++14} \\
    Expression by value & \cpp{[p=std::move(up)]} \\
    \midrule
    \structure{C++17} \\
    \cpp{this} by value & \mintinline{cpp}{[*this]} \\
    \bottomrule
  \end{tabular}
\end{frame}

\begin{frame}[fragile]
  \frametitle{Lambda return type deduction}
  \structure{C++11}
  \begin{itemize}
  \item Return type can be omitted if the body is a single return statement (or
    has no return statement).
  \item \cpp~[]() { return 3; }~ is deduced to return
    \cpp{int} (like \mintinline{cpp}{auto} does)
  \end{itemize}
  \structure{C++14}
  \begin{itemize}
  \item Return type can be omitted even if multiple statements.
  \end{itemize}
\end{frame}

\begin{frame}[fragile]
  \frametitle{\cpp{std::function}}
  \begin{itemize}
  \item Lambda expressions have anonymous types
  \item Say you want to store some in a \cpp{vector}
    \begin{itemize}
    \item They all need to be the same type!
    \end{itemize}
  \item \cpp{std::function} is a wrapper for anything that can be called
    \begin{itemize}
    \item lambdas
    \item function pointers
    \item anything with \cpp{operator()}
    \end{itemize}
  \end{itemize}
\begin{minted}{cpp}
vector<function<string(int)>> v;
v.push_back(&to_string<int>);
v.push_back([](int i) { return string(i); });
\end{minted}
\end{frame}

\begin{frame}[fragile]
  \frametitle{Generic lambdas (C++14)}
  You can use \cpp{auto} for your lambda's parameters:
\begin{minted}{cpp}
auto mul = [](auto x, auto y) { return x * y; };
int a = mul(2, 3);
double b = mul(2, 3.2);
\end{minted}
  and get a templated function call operator like this:
\begin{minted}{cpp}
struct mul_lambda {
  template<typename X, typename Y>
  auto operator()(X x, Y y) { return x * y; }
};
auto mul = mul_lambda();
\end{minted}
  This is really useful in certain circumstances (visitor pattern) but beware
overuse.
\end{frame}

\section{Move semantics}

\begin{frame}
  \frametitle{Move semantics}
  \begin{itemize}
  \item A combination of features
    \begin{itemize}
    \item rvalue references
    \item move constructors and assignment
    \end{itemize}
  \item Can speed up code by avoiding copying
  \item Can allow non-copyable objects to be transferred (see
    \cpp{unique_ptr})
  \end{itemize}
\end{frame}

\begin{frame}[fragile]
  \frametitle{Rvalue references}
  \begin{itemize}
  \item An \alert{rvalue} is kind of a temporary value
  \item In \cpp{v.push_back(string("123"))}, the string is an rvalue
  \item In contrast to an lvalue, an rvalue has no named storage location
  \item rvalues can bind to const lvalue references
  \item lvalues cannot bind to rvalue references
  \end{itemize}
\begin{minted}{cpp}
void l(int &i);  // lvalue reference
void r(int &&i); // rvalue reference

int v = 3;
l(v); // OK - lvalue to lvalue reference
r(v); // NOT OK - lvalue to rvalue reference
r(std::move(v)); // OK - rvalue ref to rvalue ref
l(3); // NOT OK - rvalue to non-const lvalue ref
r(3); // OK - rvalue to rvalue reference
\end{minted}
\end{frame}

\begin{frame}[fragile]
  \frametitle{Move construction and assignment}
  \begin{itemize}
  \item A \alert{move constructor} is like a copy constructor except the source
    is passed by rvalue reference
  \item The idea is that it gets called when the source is ``going away'' in
    some sense --- it can be destructive
  \item The other idea is that this makes the operation more efficient, e.g. by
    just transferring a pointer
  \item The \alert{move assignment operator} works on the same principle
  \end{itemize}
\begin{minted}{cpp}
struct foo {
  foo(); // default constructor
  foo(const foo&); // copy constructor
  foo(foo &&); // move constructor
  foo &operator=(const foo&); // copy assignment
  foo &operator=(foo &&); // move assignment
};
\end{minted}
\end{frame}

\section{Range-based for loop}

\begin{frame}[fragile]
  \frametitle{Range-based \cpp{for} loop}
  Instead of
\begin{minted}{cpp}
for (auto it = v.begin(); it != v.end(); ++it) {
  const string &x = *it;
}
\end{minted}
  you can now write
\begin{minted}{cpp}
for (const string &x : v) {
}
\end{minted}
  \begin{itemize}
  \item Saves typing
  \item More readable
  \item More efficient (slightly, maybe)
  \end{itemize}
\end{frame}

\section{Smart pointers}

\begin{frame}[fragile]
  \frametitle{Smart pointers}
  \begin{itemize}
  \item \cpp{auto_ptr} deprecated (removed in C++17)
    \begin{itemize}
    \item copy semantics are \alert{BROKEN}
    \item can't put it in a \cpp{vector}
    \end{itemize}
  \item \cpp{unique_ptr} is its replacement
    \begin{itemize}
    \item move-only, made possible by rvalue references
    \end{itemize}
  \item \cpp{shared_ptr} and \cpp{weak_ptr}
    \begin{itemize}
    \item \cpp{make_shared<T>(x, y, ...)} is useful
    \end{itemize}
  \end{itemize}
\end{frame}

\begin{frame}[fragile,fragile]
  \frametitle{\cpp{nullptr}}
\begin{minted}{cpp}
int x = NULL; // sure, why not?
int y = nullptr; // no way, pointers only
\end{minted}
\begin{minted}{cpp}
int foo(double x) { return 7; }
int foo(const char *s) { return s ? *s : 42; }

foo(NULL); // 7? huh?
foo(nullptr); // ahh, 42 :)
\end{minted}
\end{frame}

\section{Others}

\begin{frame}[fragile]
  \frametitle{And the rest}
  \begin{columns}
    \column{.5\textwidth}
    \begin{itemize}
    \item Initializer lists
    \item Uniform initialization
    \item \cpp|override| and \cpp{final}
    \item \cpp{enum class}
    \item Angle brackets \cpp|>>|
    \item Variadic templates
    \item Variable templates (C++14)
    \item Threading
      \begin{itemize}
      \item \cpp{std::thread}
      \item \cpp{std::mutex}
      \item \cpp{std::future}
      \item \cpp{std::async}
      \item \cpp{thread_local}
      \end{itemize}
    \end{itemize}
    \column{.5\textwidth}
    \begin{itemize}
    \item \cpp{= default} and \cpp{= delete}
    \item \cpp{static_assert}
    \item \cpp{constexpr}
    \item \cpp{long long int}
    \item \cpp{alignof} and \cpp{alignas}
    \item Tuples
    \item Hash tables
    \item Regular expressions
    \item Literals
      \begin{itemize}
      \item User-defined literals
      \item Binary literals (C++14)
      \item More literals (C++14)
      \end{itemize}
    \end{itemize}
  \end{columns}
\end{frame}

\begin{frame}
  \centering{Questions?}
\end{frame}

\appendix

\section{\appendixname}
\frame{\tableofcontents}

\subsection{Extra material}

\begin{frame}[fragile]
  \frametitle{\cpp~decltype~}
  Use to say that something has the same type as something else.
\begin{minted}{cpp}
vector<int> v = {1,2,3};
auto a = v[1]; // int
decltype(v[1]) b = v[1]; // int&
decltype(auto) c = v[1]; // int& (C++14)
\end{minted}
  Not that useful most of the time.
\end{frame}

\begin{frame}[fragile]
  \frametitle{\cpp{using} type aliases}
  \begin{itemize}
  \item More readable alternative to \cpp~typedef~
    \begin{tabular}[h]{ll}
      \toprule
      \cpp{typedef} & \mintinline{c++}{using} \\
      \midrule
      \cpp{typedef map<K,V> m} & \mintinline{c++}{using m = map<K,V>} \\
      \cpp{typedef void (*f)(int)} & \mintinline{c++}{using f = void (*)(int)} \\
      \bottomrule
    \end{tabular}
  \item Can be templated
\begin{minted}{c++}
template<typename T>
using myvector = vector<T>;
\end{minted}
  \end{itemize}
\end{frame}

\subsection{References}

\begin{frame}[fragile]
  \frametitle{References}
  \begin{itemize}
  \item \verb|https://en.wikipedia.org/wiki/C%2B%2B11|
  \item \verb|https://en.wikipedia.org/wiki/C%2B%2B14|
  \item \verb|https://en.wikipedia.org/wiki/C%2B%2B17|
  \item \verb|http://cppreference.com|
  \end{itemize}
\end{frame}

\end{document}

% Local Variables:
% TeX-command-extra-options: "-shell-escape"
% End:
