% Created 2016-06-02 Thu 16:16
\documentclass[presentation]{beamer}
\usepackage[utf8]{inputenc}
\usepackage[T1]{fontenc}
\usepackage{fixltx2e}
\usepackage{graphicx}
\usepackage{grffile}
\usepackage{longtable}
\usepackage{wrapfig}
\usepackage{rotating}
\usepackage[normalem]{ulem}
\usepackage{amsmath}
\usepackage{textcomp}
\usepackage{amssymb}
\usepackage{capt-of}
\usepackage{hyperref}
\usepackage{minted}
\usepackage{booktabs}
\usetheme{default}
\usecolortheme{}
\usefonttheme{}
\useinnertheme{}
\useoutertheme{}
\date{Auckland C++ Meetup\\14 June 2016}
\title{Recent Changes in C++}
\subtitle{Some Highlights}
\author{Toby Allsopp\\\texttt{toby@mi6.gen.nz}}

\hypersetup{
 pdfauthor={},
 pdftitle={},
 pdfkeywords={},
 pdfsubject={},
 pdfcreator={},
 pdflang={English}}
\begin{document}

\begin{frame}
 \titlepage{} 
\end{frame}
\label{sec:orgheadline31}
\begin{frame}[label={sec:orgheadline1}]{Overview}
%\setcounter{tocdepth}{9}
  \tableofcontents
\end{frame}
\section{C++11}

\subsection{Type Inference}
\begin{frame}
\tableofcontents
\end{frame}

\begin{frame}[fragile]
  \frametitle{\mintinline{c++}~auto~}
  Instead of
\begin{minted}[]{c++}
map<string, string>::iterator it = m.find("foo");
\end{minted}
  you can write
\begin{minted}[]{c++}
auto it = m.find("foo");
\end{minted}
  \begin{itemize}
  \item No more \mintinline{c++}~typedef std::map<string, string> FooBarMap~
  \item Doesn't work for member variables, function parameters (but wait for generic lambdas)
  \end{itemize}
\end{frame}

\begin{frame}[fragile]
  \frametitle{\mintinline{c++}{using} type aliases}
  \begin{itemize}
  \item More readable alternative to \mintinline{c++}~typedef~
    \begin{tabular}[h]{ll}
      \mintinline{c++}{typedef} & \mintinline{c++}{using} \\
      \midrule
      \mintinline{c++}{typedef map<K,V> m} & \mintinline{c++}{using m = map<K,V>} \\
      \mintinline{c++}{typedef void (*f)(int)} & \mintinline{c++}{using f = void (*)(int)} \\
    \end{tabular}
  \item Can be templated
\begin{minted}{c++}
template<typename T>
using myvector = vector<T>;
\end{minted}
  \end{itemize}
\end{frame}

\begin{frame}
  \frametitle{\mintinline{c++}~decltype~}
\end{frame}

\subsection{Lambda expressions}

\begin{frame}[fragile,label={sec:orgheadline4}]{Lambdas}
  Given
\begin{minted}[]{c++}
vector<int> v = { 1, 1, 2, 3 };
int limit = 1;
\end{minted}
\begin{minted}[]{c++}
auto it = find_if(v.begin(), v.end(),
  [&limit](int x) -> bool { return x > limit; });
assert(it != v.end() && *it == 2);
\end{minted}
  is equivalent to
\begin{minted}[]{c++}
struct pred {
  int &limit;
  Pred(int &limit) : limit(limit) {}
  bool operator()(int x) const { return x > limit; }
};
auto it = find_if(v.begin(), v.end(), pred(limit));
assert(it != v.end() && *it == 2);
\end{minted}
\end{frame}
\begin{frame}[fragile]{Lambdas (capturing)}
  \begin{tabular}[h]{ll}
    Nothing                 & \mintinline{c++}{[]}             \\
    Everything by reference & \mintinline{c++}~[&]~            \\
    Everything by value     & \mintinline{c++}~[=]~            \\
    Something by reference  & \mintinline{c++}{[&something]}   \\
    One of each             & \mintinline{c++}{[&byref,byval]} \\
  \end{tabular}
\end{frame}

\subsection{Move semantics}
\begin{frame}[label={sec:orgheadline6}]{move semantics}
\end{frame}
\begin{frame}
  \frametitle{rvalue reference}
\end{frame}
\begin{frame}[label={sec:orgheadline7}]{constexpr}
\end{frame}
\begin{frame}[label={sec:orgheadline8}]{initializer lists}
\end{frame}
\begin{frame}[label={sec:orgheadline9}]{uniform initialization}
\end{frame}
\begin{frame}[label={sec:orgheadline10}]{new function syntax}
\end{frame}
\begin{frame}[label={sec:orgheadline11}]{override and final}
\end{frame}
\begin{frame}[label={sec:orgheadline12}]{nullptr}
\end{frame}
\begin{frame}[label={sec:orgheadline13}]{enum class}
\end{frame}
\begin{frame}[label={sec:orgheadline14}]{Angle brackets >>}
\end{frame}
\begin{frame}[label={sec:orgheadline16}]{Variadic templates}
\end{frame}
\begin{frame}[label={sec:orgheadline17}]{String literals}
\end{frame}
\begin{frame}[label={sec:orgheadline18}]{User-defined literals}
\end{frame}
\begin{frame}[fragile,label={sec:orgheadline19}]{Threading}
 \begin{block}{\texttt{std::thread}}
\end{block}
\begin{block}{\texttt{std::mutex}}
\end{block}
\begin{block}{\texttt{std::future}}
\end{block}
\begin{block}{\texttt{std::async}}
\end{block}
\end{frame}
\begin{frame}[fragile,label={sec:orgheadline20}]{\texttt{thread\_local}}
\end{frame}
\begin{frame}[label={sec:orgheadline21}]{Defaulted and deleted special members}
\end{frame}
\begin{frame}[fragile,label={sec:orgheadline22}]{\texttt{long long int}}
\end{frame}
\begin{frame}[fragile,label={sec:orgheadline23}]{\texttt{static\_assert}}
\end{frame}
\begin{frame}[fragile,label={sec:orgheadline24}]{\texttt{alignof} and \texttt{alignas}}
\end{frame}
\begin{frame}[label={sec:orgheadline25}]{Attributes}
\end{frame}
\begin{frame}[label={sec:orgheadline26}]{Tuples}
\end{frame}
\begin{frame}[label={sec:orgheadline27}]{Hash tables}
\end{frame}
\begin{frame}[label={sec:orgheadline28}]{Regular expressions}
\end{frame}
\begin{frame}[label={sec:orgheadline29}]{Smart pointers}
\end{frame}
\begin{frame}[fragile,label={sec:orgheadline30}]{\texttt{std::function}}
\end{frame}

\section{C++14}
\label{sec:orgheadline40}
\begin{frame}[label={sec:orgheadline32}]{Function return type deduction}
\end{frame}
\begin{frame}[fragile,label={sec:orgheadline33}]{Improved \texttt{constexpr}}
\end{frame}
\begin{frame}[label={sec:orgheadline34}]{Variable templates}
\end{frame}
\begin{frame}[label={sec:orgheadline35}]{Binary literals}
\end{frame}
\begin{frame}[label={sec:orgheadline36}]{Generic lambdas}
\end{frame}
\begin{frame}[label={sec:orgheadline37}]{Lambda capture expressions}
\end{frame}
\begin{frame}[label={sec:orgheadline38}]{More literals}
\end{frame}
\begin{frame}[fragile,label={sec:orgheadline39}]{\texttt{std::make\_unique}}
\end{frame}

\section{C++17}
\label{sec:orgheadline41}
\end{document}
